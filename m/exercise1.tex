\documentclass[a4paper]{article}
\usepackage[T2A]{fontenc}
\usepackage[ngerman]{babel}
\usepackage[utf8]{inputenc}
\usepackage{csquotes}
\usepackage{amssymb}
\usepackage{amsmath}
\usepackage{float}
\usepackage{fancyhdr}
\usepackage[inline]{enumitem}
\usepackage{booktabs}
\usepackage[table]{xcolor}
\usepackage{tabularx}
%\usepackage{multirow}
%\usepackage{gensymb}
\usepackage{mathtools}
%\usepackage{titlesec}
%\usepackage{wrapfig}
\usepackage[format=plain,labelfont=it]{caption}

\usepackage{tikz}
%\usetikzlibrary{patterns}
%\usetikzlibrary{arrows.meta}
\usetikzlibrary{er,positioning}

\usepackage{pgfplots}
\pgfplotsset{compat=1.16}
\usepgfplotslibrary{fillbetween}

\usepackage{makecell}
\usepackage{graphicx}
\usepackage{multirow}

%\usepackage[a4paper, margin=2cm, top=2.5cm, bottom=2.5cm]{geometry}

\pagestyle{fancy}
\fancyhf{}
\fancyhead[C]{Mathematik}
\fancyhead[L]{Übungsblatt vom 31. März}
\fancyhead[R]{Laurenz Weixlbaumer}

\fancyfoot[C]{\thepage}

\newlist{tasks}{enumerate}{3}
\setlist[tasks]{label=(\alph*)}

\newcommand{\term}[1]{\emph{#1}}

\newcommand{\figref}[1]{\emph{Abb. \ref{#1}}}
\newcommand{\tabref}[1]{\emph{Tab. \ref{#1}}}

\newcommand{\fkey}[1]{\emph{#1}}
\newcommand{\pkey}[1]{\underline{\fkey{#1}}}

\setlength{\abovedisplayskip}{1pt}
\setlength{\belowdisplayskip}{1pt}
\setlength{\abovedisplayshortskip}{1pt}
\setlength{\belowdisplayshortskip}{1pt}

\begin{document}

\section{Gewinnschwelle}

\begin{tasks}
    \item Jenes $x_1$ für welches gilt, dass
    \begin{equation*}
        ax_1 + b = cx_1
    \end{equation*}
    ist durch einen Term in Abhängigkeit von $a$, $b$ und $c$ zu beschreiben. Dies wird erreicht durch Umformung zu
    \begin{equation*}
        x_1(a - c) = -b
    \end{equation*}
    und weiters
    \begin{equation}\label{eq:task1_sol}
        x_1 = \frac{-b}{a - c} \quad\text{für}\quad a \neq c
    \end{equation}
    wobei \eqref{eq:task1_sol} der gesuchte Term ist.

    \item Parameter $a$ beschreibt die Produktionskosten einer Einheit wobei $b$ konstante, von der Anzahl der prod. Einheiten unabhängige, Kosten (z. B. das Anblasen eines Hochofens) beschreibt. $c$ beschreibt den Preis einer verkauften Einheit.

    Soll \eqref{eq:task1_sol} als Funktion $f(d)$ für $d = c - a$ interpretiert werden, ergibt sich der Funktionsterm
    \begin{equation}\label{eq:task1_f}
        f(d) = \frac{-b}{a - c} = \frac{b}{d}
    \end{equation}
    der eine Hyperbel wie sie in \figref{fig:task1_f} dargestellt ist beschreibt.

    \begin{figure}[!h]
        \centering
        \begin{tikzpicture}
            \begin{axis}%
            [
                grid=major,
                xmin=0,
                xmax=10,
                ymin=0,
                ymax=10,
                xlabel={$d$},
                ylabel={$f(d)$},
                ylabel style={rotate=-90},
                samples=400,
                domain=0:10,
            ]
                \addplot[thick] (x,{1/x});
            \end{axis}
        \end{tikzpicture}
        \caption{Funktionsgraph von \eqref{eq:task1_f} für $b = 1$}
        \label{fig:task1_f}
    \end{figure}
\end{tasks}

\section{Wirkstoffkonzentration}

\begin{tasks}
    \item Die Gleichung $c(t_1) = c(0) \cdot 0,7$ sagt im gegeben Kontext aus, dass $t_1$ Stunden nach der Einnahme 70\% des Wirkstoffes im Blut verweilen.

    Weiters impliziert sie ausgehend von der Gleichung
    \begin{equation}\label{eq:task2_oeq}
        c(t) = c(0) \cdot 0,85^t
    \end{equation}
    dass $0,7 = 0,85^{t_1}$ woraus sich $t_1 = 2,1947$ ergibt.

    \item Aus dem Graph in \figref{fig:task2_f} ist abzulesen, dass die absolute Änderung in den Intervallen mit steigendem $t$ sinkt. Die relative Änderung bleibt jedoch konstant.

    \begin{figure}[!h]
        \centering
        \begin{tikzpicture}
            \begin{axis}%
            [
                grid=minor,
                xlabel={$t$},
                xmin=0,
                xmax=10,
                ymin=0,
                ylabel={$c(t)$},
                samples=400,
                domain=0:10,
                extra y ticks={100,70,49,34.3},
                extra y tick style={%
                    grid=major,
                    ticklabel pos=top
                },
                extra x ticks={2.1947,4.3894,6.5841},
                extra x tick labels={$t_1$, $2t_1$, $3t_1$},
                extra x tick style={%
                    grid=major,
                    ticklabel pos=top
                },
                ylabel style={rotate=-90}
            ]
                % 100, 70, 49, 34.3 (2.1947)

                \addplot[thick] (x,{100*0.85^x});

                \draw [|-|] (2.1947,100) --node[right]{30} (2.1947,70);
                \draw [|-|] (2 * 2.1947,70) --node[right]{21} (4.3894, 49);
                \draw [|-|] (6.5841, 49) --node[right]{14,7} (6.5841, 34.3);
            \end{axis}
        \end{tikzpicture}
        \caption{Funktionsgraph von \eqref{eq:task2_oeq} für $c(0) = 100$}
        \label{fig:task2_f}
    \end{figure}
\end{tasks}

\section{Polynomfunktion vierten Grades}

\begin{tasks}
    \item Die erste Ableitung der Funktion $f(x) = ax^4 + x^2 + c$ für $a = 1$ ist
    \begin{equation*}
        f'(x) = 4x^3 + 2x
    \end{equation*}
    woraus die Steigung der Funktion an der Stelle $x = 2$ mit $f'(2) = 36$ berechnet werden kann. Die Angabe von $c$ ist nicht erforderlich nachdem die Ableitung einer von der Variable unabhängigen Konstanten immer null ergibt.

    \item Die zweite Ableitung von $f$ ist
    \begin{equation*}
        f''(x) = a \cdot 12x^2 + 2
    \end{equation*}
    und hat für $a > 0$ ersichtlicherweise keine Nullstellen. Es gibt also keinen Punkt in $f$ an dem der Graph keine Krümmung hat, demzufolge kann $f$ keine Wendestellen haben.
\end{tasks}

\section{Lineare Funktionen}

\begin{tasks}
    \item Ist $g$ als
    \begin{equation}\label{eq:task4_gx}
        g(x) = -0.5x + 6
    \end{equation}
    definiert, gilt also $a = -0.5$, so ist die Gleichung $f(x) = g(x)$ nicht lösbar, wie auch in \figref{fig:task4_f} dargestellt.

    \begin{figure}[!h]
        \centering
        \begin{tikzpicture}
            \begin{axis}
            [
                grid=major,
                xlabel={$x$},
                ylabel={$f(x), g(x)$},
                samples=100,
                domain=0:20,
                xmin=0,
                xmax=20,
                legend entries={$f(x)$,$g(x)$}
            ]

                \addplot[thick] {-0.5 * x + 3};
                \addplot[dashed] {-0.5 * x + 6};
            \end{axis}
        \end{tikzpicture}
        \caption{Funktionsgraph von $f$ und \eqref{eq:task4_gx}}
        \label{fig:task4_f}
    \end{figure}

    \item In Abhängigkeit von $a$ kann die Gleichung $f(x) = g(x)$ als
    \begin{equation*}
        s(a) = -\frac{6}{2a+1} \quad\text{für}\quad a \neq -0,5
    \end{equation*}
    dargestellt werden.

    Um $f$ in einem Punkt $> 0$ zu schneiden, muss $g$ eine kleinere Steigung als $f$ haben. Demnach ist $s$ ist für alle $a < -0.5$ positiv.
\end{tasks}

\section{Würfel}

\begin{tasks}
    \item Alle Elemente eines \emph{Grundraums} $\Omega$ sind Elementarereignisse $\omega$, die als mögliches Ergebnis eines Zufallsexperiments angesehen werden können. Ein \emph{Ereignis} ist eine Teilmenge des Grundraums.

    Im gegebenen Beispiel ist der Grundraum $A = \{1, 2, 3, 4, 5, 6\}$ und die Ereignismenge $B = \{2, 3, 5\}$.

    Nachdem $7 \not\in A$ ist $B = \emptyset$; $B$ ist ein \emph{Nullereignis}.

    \item Die Wahrscheinlichkeit, mindestens einmal eine gerade Zahl zu würfeln kann durch die Gegenwahrscheinlichkeit des Ereignisses \enquote{niemals eine gerade Zahl gewürfelt}, also
    \begin{equation}\label{eq:task5_eq}
        P = 1 - \left( \frac{1}{2} \right)^n
    \end{equation}
    ausgedrückt werden.

    Durch Gleichsetzung von \eqref{eq:task5_eq} mit $0,99$ kann das kleinste $n$ für eine $P \geq 0.99$ ermittelt werden. Nach
    \begin{align*}
        0,99 &= 1 - \left( \frac{1}{2} \right) ^n\\
        n &= 6,6439
    \end{align*}
    muss der Würfel also mindestens $7$ Mal geworfen werden.

\end{tasks}

\end{document}
